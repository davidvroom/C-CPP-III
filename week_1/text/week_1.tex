\documentclass[12pt]{article}
\usepackage[left=1in, right=1in]{geometry}

\usepackage{url}

%\usepackage{arydshln}

\usepackage{graphicx}

\usepackage{color}
\definecolor{light-gray}{gray}{0.30}

\usepackage{verbatim}

\usepackage{listings}
\lstset{
	frame=leftline,
	frameround=ttff,
	numbers=left,
	language=C++,
	showstringspaces=False,
	extendedchars=False,
	numberstyle=\footnotesize,
	basicstyle=\small\ttfamily,
	commentstyle=\color{light-gray}\slshape,
	belowskip=1.5em,
	aboveskip=1.5em,
	fontadjust,
	tabsize=4,		%added for proper tab alligning
	xleftmargin=0cm,
	xrightmargin=0cm
}


\newcommand{\desc}[1]{\textit{#1} \vspace{1em}}

\title{\itshape Exercises week 1: Function Templates}

\author{
	Klaas Isaac Bijlsma \\ s2394480
	\and
	David Vroom \\ s2309939
}

\date{\today}

\begin{document}
\maketitle

\section*{Exercise 1}
\desc{Show that templates don't result in `code bloat'}

A function template \texttt{add} and a union \texttt{PointerUnion} were defined in separate header files. We use this union to print the address of the function \texttt{add}. There are two source files, one for \texttt{fun} and one for \texttt{main}. The function \texttt{fun}, which includes \texttt{add.h}, instantiates \texttt{add} for \texttt{int}s and prints its address. Then, in \texttt{main} the same happens and \texttt{fun} is called. When the two source files of \texttt{fun} and \texttt{main} are compiled to object modules, they both contain an instantiation of \texttt{add}. Then they are linked to obtain an executable. The output of this executable gives two identical addresses, which means that only one instantiation of \texttt{add} is present. So the linker prevents 'code bloat'.


\lstinputlisting[title=\texttt{add.h}]{../ex1/add.h}
\lstinputlisting[title=\texttt{pointerunion.h}]{../ex1/pointerunion.h}
\lstinputlisting[title=\texttt{fun.cc}]{../ex1/fun.cc}
\lstinputlisting[title=\texttt{main.cc}]{../ex1/main.cc}

\clearpage

\section*{Exercise 2}
\desc{}


\clearpage

\section*{Exercise 3}
\desc{}


\clearpage

\section*{Exercise 4}
\desc{}


\clearpage

\section*{Exercise 5}
\desc{}


\clearpage

\end{document}
