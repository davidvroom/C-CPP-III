\documentclass[12pt]{article}
\usepackage[left=1in, right=1in]{geometry}

\usepackage{url}

%\usepackage{arydshln}

\usepackage{graphicx}

\usepackage{color}
\definecolor{light-gray}{gray}{0.30}

\usepackage{verbatim}

\usepackage{listings}
\lstset{
	frame=leftline,
	frameround=ttff,
	numbers=left,
	language=C++,
	showstringspaces=False,
	extendedchars=False,
	numberstyle=\footnotesize,
	basicstyle=\small\ttfamily,
	commentstyle=\color{light-gray}\slshape,
	belowskip=1.5em,
	aboveskip=1.5em,
	fontadjust,
	tabsize=4,		%added for proper tab alligning
	xleftmargin=0cm,
	xrightmargin=0cm
}


\newcommand{\desc}[1]{\textit{#1} \vspace{1em}}

\title{\itshape Exercises week 1: Function Templates}

\author{
	Klaas Isaac Bijlsma \\ s2394480
	\and
	David Vroom \\ s2309939
}

\date{\today}

\begin{document}
\maketitle

\section*{Exercise 1}
\desc{Show that templates don't result in `code bloat'}

A function template \texttt{add} and a union \texttt{PointerUnion} were defined in separate header files. We use this union to print the address of the function \texttt{add}. There are two source files, one for \texttt{fun} and one for \texttt{main}. The function \texttt{fun}, which includes \texttt{add.h}, instantiates \texttt{add} for \texttt{int}s and prints its address. Then, in \texttt{main} the same happens and \texttt{fun} is called. When the two source files of \texttt{fun} and \texttt{main} are compiled to object modules, they both contain an instantiation of \texttt{add}. Then they are linked to obtain an executable. The output of this executable gives two identical addresses, which means that only one instantiation of \texttt{add} is present. So it can be concluded that the linker prevents 'code bloat'.


\lstinputlisting[title=\texttt{add.h}]{../ex1/add.h}
\lstinputlisting[title=\texttt{pointerunion.h}]{../ex1/pointerunion.h}
\lstinputlisting[title=\texttt{fun.cc}]{../ex1/fun.cc}
\lstinputlisting[title=\texttt{main.cc}]{../ex1/main.cc}

\clearpage


\section*{Exercise 2}
\desc{Learn to embed a function template in a function template}

We used the following code,

\lstinputlisting[title=\texttt{as.h}]{../ex2/as.h}
\lstinputlisting[title=\texttt{main.cc}]{../ex2/main.cc}

\clearpage


\section*{Exercise 3}
\desc{Learn to construct a generic index operator}

We used the following code,

\lstinputlisting[title=\texttt{storage.h}]{../ex3/storage/storage.h}

\clearpage


\section*{Exercise 4}
\desc{Learn to design and use a function template}

The code below is based on the solution of exercise 48 of part II of the C++ course.

\lstinputlisting[title=\texttt{exception/exception.h}]{../ex4/exception/exception.h}
\lstinputlisting[title=\texttt{exception/exception.ih}]{../ex4/exception/exception.ih}
\lstinputlisting[title=\texttt{exception/what.cc}]{../ex4/exception/what.cc}
\lstinputlisting[title=\texttt{main.cc}]{../ex4/main.cc}

\clearpage

\section*{Exercise 5}
\desc{Learn to design a generic function template}

We used the following code,

\lstinputlisting[title=\texttt{forwarder/forwarder.h}]{../ex5/forwarder/forwarder.h}
\lstinputlisting[title=\texttt{main.cc}]{../ex5/main.cc}


\clearpage

\section*{Exercise 7}
\desc{Gain some experience with the function selection mechanism}

\lstinputlisting[title=\texttt{source}]{../ex7/opdracht.txt}

\textbf{Why is the scope resolution operator required when calling \texttt{max()}?}\\
Apparantly, there is another function template for a function \texttt{max} in the header file \texttt{iostream}, that also expects two arguments that are a \texttt{const \&} to the same formal type. The function selection mechanism will find a draw between this template function and ours on all criteria and therefore end the process with an ambiguity. To specify that we call the function \texttt{max} for which we defined a template above \texttt{main}, we need the scope resolution operator.

\textbf{When compiling this function the compiler complains with a message like:
\texttt{max.cc:13: error: no matching function for call to 'max(double, int)'}    
Why doesn't the compiler generate a \texttt{max(double, double)} function in this case?}\\
The standard conversion from \texttt{int} to \texttt{double} is only allowed for template non-type parameters. It is not part of the three allowed types of parameter type transformations. Since we deal with template type parameters, it is not possible.

\textbf{Assume we add a function 
\texttt{double max(double const \&left, double const \&right)}
to the source. Explain why this solves the problem.}\\
Now we have added a normal function (not a function template), for which the compiler is allowed to make the implicit conversion from \texttt{int} to \texttt{double} to fit the arguments to the parameters.

\textbf{Assume we would then call \texttt{::max('a', 12)}. Which \texttt{max()} function is then used and why? }\\
Now again the normal, non-template function is used. Both arguments are converted to a \texttt{double}.

\textbf{Remove the additional \texttt{max} function. Without using casts or otherwise changing the argument list of the function call \texttt{max(3.5, 4)}, how can we get the compiler to compile the source properly? }\\
By calling \texttt{::max<double>(3.5, 4)}.

\textbf{Specify a general characteristic of the answer to the previous question (i.e., can the approach always be used or are there certain limitations?). }\\
This only works if a standard conversion exists to convert the arguments to the type that is specified between pointy brackets. 







\clearpage

\end{document}
