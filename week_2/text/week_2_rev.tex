\documentclass[12pt]{article}
\usepackage[left=1in, right=1in]{geometry}

\usepackage{url}

%\usepackage{arydshln}

\usepackage{graphicx}

\usepackage{color}
\definecolor{light-gray}{gray}{0.30}

\usepackage{verbatim}

\usepackage{listings}
\lstset{
	frame=leftline,
	frameround=ttff,
	numbers=left,
	language=C++,
	showstringspaces=False,
	extendedchars=False,
	numberstyle=\footnotesize,
	basicstyle=\small\ttfamily,
	commentstyle=\color{light-gray}\slshape,
	belowskip=1.5em,
	aboveskip=1.5em,
	fontadjust,
	tabsize=4,		%added for proper tab alligning
	xleftmargin=0cm,
	xrightmargin=0cm
}


\newcommand{\desc}[1]{\textit{#1} \vspace{1em}}

\title{\itshape Exercises week 2: Class Templates - Revision}

\author{
	Klaas Isaac Bijlsma \\ s2394480
	\and
	David Vroom \\ s2309939
}

\date{\today}

\begin{document}
\maketitle

\section*{Exercise 9}
\desc{Learn to design a member template'}

\textbf{In the first attempt we forgot to decrement \texttt{d\_nAvailable} when required.}

We used the following code,

\lstinputlisting[title=\texttt{semaphore.h}]{../ex9/semaphore.h}

\clearpage


\section*{Exercise 13}
\desc{Learn to create a generic constructor for a virtual base class}

\textbf{In the first attempt our return statement of the \texttt{make} function was incorrect. We added \texttt{std::move}.}

We used the following code,

\lstinputlisting[title=\texttt{abc/abc.h}]{../ex13/abc/abc.h}
\lstinputlisting[title=\texttt{abc/destructor.cc}]{../ex13/abc/destructor.cc}
\lstinputlisting[title=\texttt{abc/interface.cc}]{../ex13/abc/interface.cc}
\lstinputlisting[title=\texttt{derived1/derived1.h}]{../ex13/derived1/derived1.h}
\lstinputlisting[title=\texttt{derived1/derived1.ih}]{../ex13/derived1/derived1.ih}
\lstinputlisting[title=\texttt{derived1/derived1.cc}]{../ex13/derived1/derived1.cc}
\lstinputlisting[title=\texttt{derived1/run.cc}]{../ex13/derived1/run.cc} 
\lstinputlisting[title=\texttt{derived2/derived2.h}]{../ex13/derived2/derived2.h}
\lstinputlisting[title=\texttt{derived2/derived2.ih}]{../ex13/derived2/derived2.ih}
\lstinputlisting[title=\texttt{derived2/derived2.cc}]{../ex13/derived2/derived2.cc}
\lstinputlisting[title=\texttt{derived2/run.cc}]{../ex13/derived2/run.cc} 
\lstinputlisting[title=\texttt{process/process.h}]{../ex13/process/process.h}
\lstinputlisting[title=\texttt{main.cc}]{../ex13/main.cc}   

\clearpage

\section*{Exercise 14}
\desc{Learn to add iterators to a class}

\textbf{This is the first attempt.}

Since the iterators have to be passed to the \texttt{sort} generic algorithm, the iterators are of iterator type \texttt{random\_access\_iterator}.

We used the following code,

\lstinputlisting[title=\texttt{storage.h}]{../ex14/storage.h}


\clearpage


\end{document}
